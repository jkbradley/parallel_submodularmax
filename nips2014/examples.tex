\section{Illustrative examples}

The following examples illustrate how (i) the simple (uni-directional) greedy algorithm may fail for non-monotone submodular functions, and (ii) where the coordination-free double greedy algorithm can run into trouble.

\subsection{Greedy and non-monotone functions}\label{app:greedyfail}

For illustration, consider the following toy example of a non-monotone submodular function. We are given a ground set $V = \{v_0, v_1, v_2, \ldots, v_k\}$ of $k+1$ elements, and a universe $U = \{u_1, \ldots, u_k\}$. Each element $v_i$ in $V$ covers elements $\mathrm{Cov}(v_i) \subseteq U$ of the universe. In addition, each element in $V$ has a cost $c(v_i)$. We are aiming to maximize the submodular function
\begin{equation}
  \label{eq:1}
  F(S) = |\bigcup_{v \in S}\mathrm{Cov}(v)| - \sum_{v \in S}c(v).
\end{equation}
Let the costs and coverings be as follows:
\begin{align}
  \mathrm{Cov}(v_0) &= U \qquad c(v_0) &= k-1\\
  \mathrm{Cov}(v_i) &= u_i  \qquad c(v_i) &= \epsilon \ll 1/k \text{ for all } i > 0.
\end{align}
Then the optimal solution is $S^* = V\setminus v_0$ with $FS^*) = k - k\epsilon$. 

The greedy algorithm of \citet{nemhauser1978} always adds the element with the largest marginal gain. Since $F(v_0) = 1$ and $F(v_i) = 1-\epsilon$ for all $i > 0$, the algorithm would pick $v_0$ first. After that, any additional element only has a negative marginal gain, $F(\{v_0,v_i\}) - F(v_0) = - \epsilon$. Hence, the algorithm would end up with a solution $F(v_0) = 1$ or worse.

For the double greedy algorithm, the scenario would be the following.
If $v_0$ happens to be the first element, then it picks it with probability $p(v_0) = \frac{1}{1 + k-1} = \frac{1}{k}$. If it picks $v_0$, nothing else will be added afterwards. If it does not pick $v_0$, then any other element has a probability of $\frac{1-\epsilon}{1-\epsilon} = 1$ of being added.
If $v_0$ is not the first element, then any element before $v_0$ is added with probability $p(v_i) = 1-\epsilon$, and as soon as an element $v_i$ has been picked, $v_0$ will not be added any more. Hence, with high probability, this algorithm returns the optimal solution. The deterministic version surely does.


\subsection{When is coordination needed?}

